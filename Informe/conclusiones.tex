Aca escribimos las conclusiones.
¿Como afecta la calibracion del sistema en el resto de las etapas?
La calibracion es la parte inicial y fundamental para la resolucion de la reconstruccion 3D, ya que en ella obtenemos las direcciones de fuente de luz que afectara directamente al resultado, independientemente de que tan bueno sea el algoritmo para resolver sistemas lineales o estimar valores.

¿Como impacta la eleccion de las 3 direcciones de iluminacion para el calculo de las normales?
la eleccion de las direcciones de iluminacion afectara directamente a las normales ya que se aproximan apartir de ellas y si la matriz esta mal condicionada dara resultados no muy buenos.

Dado que cada imagen RGB tiene 3 canales(rojo,verde y azul), ¿Como pueden ser combinados para afectar lo menos posible los resultados?
La vision humana tiende a diferenciar mas escalas de verde lo cual hace que al hacer un promedio ponderado para el color verde se obtengan mejores resultados(lo cuales no pudimos verlos en las imagenes que probamos)

//¿Como afectan la estimacion de las profundidades el calculo de las normales?


//¿Que metodos de solucion de los sistemas lineales arrojan mejores calculos de profundidades?
