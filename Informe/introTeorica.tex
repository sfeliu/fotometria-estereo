El foco de este trabajo practico es la resolucion de sistmas de ecuaciones lineales. La idea principal de la resolucion de dichos sistemas es despejar todas sus variables. En este trabajo pretendemos resolver sistemas lineales que tienen muchas incognitas lo cual hace que sea impractico intentar resolverlos en lapiz y papel. En este trabajo utilizamos multiples metods numericos que simplifican la tarea de encontrar la solucion de estos sistemas. El metodo mas comun es el de eliminacion gaussiana que consiste en aplicar tres operaciones basicas una cantidad finita de veces para transormar al sistema de ecuaciones en otro equivalente(traingula el sistema) donde despejar las incognitas es una tarea trivial utilizando "backwards substitution". Estas tres operaciones basicas son multiplicar una fila por un escalar, sumar o restar un multiplo de una fila con otra y intercambiar filas. Siempre teniendo en cuenta que una fila no denota solo los coeficientes de las variables de una ecuacion sino tambien su termino independiente.\par
\indent Aunque el metodo de eliminacion gaussiana es muy comun y bastante simple de implementar su complejidad temporal es O($n^{3}$) y cuando tenemos que resolver multiples sistema de ecuaciones lineales esta complejida no es tan buena. Para estos casos, si las ecuaciones de los sistemas lineales tienen coeficientes identicos pero distintos terminos independientes, existe el metodo de factorizacion LU donde la L es una matriz triangular inferior con unos es su diagonal y U es una matriz triangular superior. Apesar de que encontrar la factorizacion LU de una matriz tiene costo O($n^{3}$), la resolucion del sistema A$x$ = LU$x$ = b tiene costo O($n^{2}$), lo cual hace que en las situaciones donde hay que resolver muchas veces el sistema Ax = b para diferentes b sea mucho mas eficiente que utilizar el algoritmo de eliminacion de gauss. La demostracion de porque los algoritmos de eliminacion guassiana y el de factorizacion LU tienen complejidad O($n^{3}$) y O($n^{2}$) respectivamente, se puede ver en el libro Numerical Analysis de Burden. \par
\indent La existencia de la factorizacion LU no esta garantizada para cualquier matriz. Dada una matriz inversible, las condiciones necesarias y suficientes para que exista la factorizacion LU son que no aparezca un zero como pivote a lo largo de la factorizacion LU o que todas las submatrices principales sean inversibles.\par
Aunque la factorizacion LU no siempre existe, la factorizacion PLU si. Esta factorizacion consiste de tres matrices cuadradas P , L y U donde L es una matriz triangular inferior y U es  una matriz triangular superior.  La matriz P es designada una matriz de permutacion que es necesaria cuando al aplicar la eliminacion de gauss hace falta hacer intercambios de filas. A diferencia de la facotrizacion LU, donde la L tiene que tener unos en la diagonal, la L de la PLU puede no tener todos unos es su diagonal.\par
\indent Un problema que no debe pasar desapercibido cuando trabajamos con operaciones aritmeticas en la computadora es el de perdida de digitos significativos. Esto ocurre porque la computadora trabaja con aritmetica de digitos finitos lo cual quiere decir que la representacion de los numeros en la computadora tienen una cantidad de digitos finitos para ser representados. Los errores mas comunes son el error de redondeo, el de cancelacion catastrofica y el error que obtenemos al dividir por un numero muy pequeno. El error de redondeo ocurre cuando trabajamos con un numero cuya representacion requiere mas digitos de los que disponemos en la computadora. El error de cancelacion catastrofica surge cuando restamos dos numeros que son tan parecidos que se se pierden demaciados digitos significativos. El problema que ocurre con una division donde el denominador es mucho mas pequeno que el numerador es que si el numerador tiene acumulado errores de operaciones anterirores la division por un numero mucho mas chico va a propagar esos errores.\par
\indent En los algoritmos que nosotros implementamos abordar este problema es imprescindible ya que al computar el algoritmo de eliminacion de gauss se hacen mas de n3 sumas, restas, diviciones y multiplicaciones, donde n es la cantidad de filas y columnas de la matriz asociada al sistema. La mejor forma que nosotros encontramos de acotar este tipo de errores es implementado el metodo de pivoteo parcial en nuestro algoritmo de eliminacion gaussiana. El pivoteo parcial consiste en que para cada iteracion del algoritmo el pivote sea el elemento mayor de la columna en la que se encuentra.\par
\indent Otra factorizacion de matrices que utilizamos es la factorizacion $LL^{t}$, conocida como factorizacion de cholesky. Hay dos condiciones que tiene que complir una matriz para garantizar la existencia de esta factorizacion. Primero que sea simetrica y segundo que sea definida-positiva. Se dice que una matriz es definida positiva cuando $x^{t}Ax$ $>$ 0 $\forall$ x $\in$ ${\rm I\!R}^{nxn}$.Una propiedad importante para destacar que decir que una matriz es definida-positiva es equivalente a decir que tiene factorizacion LU y que todos los elementos de su diagonal son positivos. Cuando aplicable, esta factorizacion resulta ser bastante mas eficiente que la factorizacion LU.\par
\indent La estabilidad numerica de un sistema lineal es un concepto sumamente importante en el momento de conciderar  el error relativo que tiene la solucion que obtenemos del sistema. Para ver si un sistema es estable hablamos del condicionamiento del sistema. A esta idea la cuantificamos utilizando el numero de condicion $\kappa$(A) = $||$A$||$ $||A^{-1}||$. Se dice que un sistema lineal esta bien condicionado si el numero esta lo mas cerca posible de 1 y mal condicionado si es mucho mas grande que 1. Aunque no hay ningun criterio exacto para saber cual es el punto de quiebre entre mal y bien condicionado, segun nuestras clases teoricas de la materia, si el numero de condicion es mayor a 10,000, usualmente se puede decir que el sistema esta mal condicionado.\par