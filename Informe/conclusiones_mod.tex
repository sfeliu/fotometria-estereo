

Las conclusiones finales de este trabajo desde principio a fin son las siguientes:
\begin{itemize}
\item Al momento de la calibración de la imágen es fundamental tomar bien las medidas, ya que si se arrastra este error, a medida que se le aplican los métodos se potencia.
\item Se debe seleccionar de buena manera las direcciones de luz para eliminar las mayores posibilidades de tener algún error de redondeo al medir las normales.
\item Al momento de calcular las profundidades se debe ser inteligente y poder rebuscarse para no consumir tanta memoria, ya que se trabajan con matrices demasiado grandes para tenerlas en memoria. En este paso es fundamental trabajar con matrices esparzas ya que poseen una gran cantidad de ceros.
\end{itemize}